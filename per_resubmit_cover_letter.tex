% $Id: for_president.tex,v 1.1 2008/11/24 13:14:13 gkapfham Exp $

\documentclass[11pt]{article}

%\documentclass{book}
%\usepackage{phic}

%\usepackage{times}

\usepackage[T1]{fontenc}
\usepackage{mathptmx}

%\makeindex

% Settings that I normally use
%\textwidth = 6.5in
%\textheight = 9.00in
\topmargin 0.0in
%\oddsidemargin 0.0in
%\evensidemargin 0.0in 

% settings from Allen Tucker 650 --> 640
\setlength{\textwidth} {420pt}
\setlength{\textheight} {620pt} 
\setlength{\oddsidemargin} {20pt}
\setlength{\marginparwidth} {72in}

%\usepackage{times}
%\usepackage{mathptm}

% this allows us to get the definitions that are not italicized

\usepackage{amsthm}

\usepackage{amsmath}
\usepackage{amsfonts}

\usepackage{url}

%\usepackage[named]{algo}
%\usepackage{algo}

\usepackage{listings}
%\usepackage{apalike}
\usepackage[authoryear]{natbib}
%\usepackage{chicago}
%\usepackage{keyval}
%\usepackage{epsfig} 
\usepackage{array}
%\usepackage{fancyhdr} 

% set it so that subsubsections have numbers and they
% are displayed in the TOC (maybe hard to read, might want to disable)

\setcounter{secnumdepth}{3}
\setcounter{tocdepth}{3}

% define widow protection

\def\widow#1{\vskip #1\vbadness10000\penalty-200\vskip-#1}

\clubpenalty=10000  % Don't allow orphans
\widowpenalty=10000 % Don't allow widows

% this should give me the ability to use some math symbols that 
% were available by default in standard latex (i.e. \Box)

\usepackage{latexsym}

% define a little section heading that doesn't go with any number

\def\littlesection#1{
\widow{2cm}
\vskip 0.5cm
\noindent{\bf #1}
\vskip 0.0001cm 
%\noindent
}

%Check if we are compiling under latex or pdflatex
%% \ifx\pdftexversion\undefined
%%   \usepackage[dvips]{graphics}
%% \else
%%   \usepackage[pdftex]{graphics}
%% \fi

%\lhead{Section \thesection}
%\chead{}
%\rhead{Page \thepage}
%\lfoot{Gregory M. Kapfhammer}
%\cfoot{}
%\rfoot{\today} 

%% \pagestyle{fancyplain}

%% \newcommand{\tstamp}{\today}   
%% %\renewcommand{\chaptermark}[1]{\markboth{#1}{}}
%% \renewcommand{\sectionmark}[1]{\markright{#1}}
%% \lhead[\Section \thesection]            {\fancyplain{}{\rightmark}}
%% \chead[\fancyplain{}{}]                 {\fancyplain{}{}}
%% \rhead[\fancyplain{}{\rightmark}]       {\fancyplain{}{\thepage}}
%% %\lfoot[\fancyplain{}{}]                 {\fancyplain{\tstamp}{\tstamp}}
%% \cfoot[\fancyplain{\thepage}{}]         {\fancyplain{\thepage}{}}
%% %\rfoot[\fancyplain{\tstamp} {\tstamp}]  {\fancyplain{}{}}

%% \newlength{\myVSpace}% the height of the box
%% \setlength{\myVSpace}{1ex}% the default, 
%% \newcommand\xstrut{\raisebox{-.5\myVSpace}% symmetric behaviour, 
%%   {\rule{0pt}{\myVSpace}}%
%% }

% A paraphrase mode that makes it easy to see the stuff that shouldn't
% stay in for the final paper

\newdimen\tmpdim
\long\def\paraphrase#1{{\parskip=0pt\hfil\break
\tmpdim=\hsize\advance\tmpdim by -15pt\noindent%
\hbox to \hsize
{\vrule\hskip 3pt\vrule\hfil\hbox to \tmpdim{\vbox{\hsize=\tmpdim
\def\par{\leavevmode\endgraf}
\obeyspaces \obeylines
\let\par=\endgraf
\bf #1}}}}}

% leave things with no spacing extra spacing in the final version of the paper
\renewcommand{\baselinestretch}{1.0}    % must go before the begin of doc

% suppress the use of indentation for a paragraph

\setlength{\parindent}{0.0in}
\setlength{\parskip}{0.1in}

\begin{document}

%\baselineskip=22pt
%\setlength{\parindent} {.5in}

\newtheorem{assume}{Assumption}

\newtheorem{consider}{Consideration}

% we are now going with the new method that is from the DIATOMS proposal

%\newtheorem{problem}{Problem}
%\newtheorem{define}{Definition}

\theoremstyle{definition}
\newtheorem{definition}{Definition}

\theoremstyle{definition}
\newtheorem{problem}{Problem}

\theoremstyle{definition}
\newtheorem{principle}{Principle}

% handle widows appropriately
\def\widow#1{\vskip #1\vbadness10000\penalty-200\vskip-#1}

% build the title section

\makeatletter

\def\maketitle{%
  %\null
  \thispagestyle{empty}%
  %\vfill
  \begin{center}%\leavevmode
    %\normalfont
    {\Huge \@title\par}%
    %\hrulefill\par
    {\normalsize \@author\par}%
    \vskip .4in
%    {\Large \@date\par}%
  \end{center}%
  %\vfill
  %\null
  %\cleardoublepage

  }

\makeatother

% build the author section
Gregory M. Kapfhammer\\
Department of Computer Science\\
Allegheny College \\
October 27, 2010

\noindent
To Whom it May Concern,

My co-authors and I would like to thank the editor and past-editor of
{\em Performance Evaluation Review} and the reviwers for enabling us
to make substantial improvements to our paper.  This cover letter
outlines the changes that we made in response to the comments from the
two reviewers.

Reviewer 1: \vspace*{-.15in}
\renewcommand{\labelitemi}{$\rightarrow$}

\begin{enumerate}

  \item {\bf List of Contributions}: The paper contains a list of
    contributions that is too long.

    \begin{itemize}
      
      \item We have substantially reduced the number of items listed
        as contributions.  For instance, all of the points about the
        experimental results have now been removed and placed into the
        appropriate location in Section 5.

    \end{itemize}

  \item {\bf Benchmark Naming Conventions}: The names for the
    benchmarks are misleading because the ``macro'' benchmarks are
    only slightly bigger than the ``micro'' benchmarks.

    \begin{itemize}
      
      \item Section 2.2 now refers to ``nano'' and ``micro''
        benchmarks, with Footnote 1 justifying our choice of this
        terminology.  All of the text and graphs have been updated to
        reflect this new naming convention.

    \end{itemize}

  \item {\bf Choice of Communication Primitives}: All of the chosen
    communication primitives require access to the network interface.
    Wouldn't it be possible to use other, perhaps more efficient,
    primitives that can avoid accessing the network?

    \begin{itemize}
      
      \item Section 1 contains a revised introduction that better
        justifies our focus on remote communication primitives in the
        Java programming language.  In essence, many of the custom
        communication primitives can incur increases in source code
        complexity and implementation effort.  This empirical study in
        this paper in the first step towards determining if
        easy-to-program remote communication primitives in the Java
        language could be used instead of custom schemes.  Moreover,
        the benchmarking framework in this paper could also be used to
        evaluate other types of communication primitives.  We have
        revised Sections 1, 6, and 7 to reflect these facts.

    \end{itemize}

  \item {\bf Minor Changes to Graphs and Sentences}: The fonts in
    Figures 3, 4, and 5 are too small and several of the sentences are
    difficult to understand.

    \begin{itemize}
      
      \item In order to ensure that the paper fits into the ten page
        limit, we cut many paragraphs of content.  However, space
        constraints currently prevent us from substantially increasing
        the font size in the graphs.  Finally, we have revised many of
        the sentences in the paper in order to guarantee that the
        content is easy to understand.

    \end{itemize}

\end{enumerate}

\end{document}
