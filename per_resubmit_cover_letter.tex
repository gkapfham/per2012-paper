% $Id: for_president.tex,v 1.1 2008/11/24 13:14:13 gkapfham Exp $

\documentclass[11pt]{article}

%\documentclass{book}
%\usepackage{phic}

%\usepackage{times}

\usepackage[T1]{fontenc}
\usepackage{mathptmx}

%\makeindex

% Settings that I normally use
%\textwidth = 6.5in
%\textheight = 9.00in
\topmargin 0.0in
%\oddsidemargin 0.0in
%\evensidemargin 0.0in

% settings from Allen Tucker 650 --> 640
\setlength{\textwidth} {420pt}
\setlength{\textheight} {620pt}
\setlength{\oddsidemargin} {20pt}
\setlength{\marginparwidth} {72in}

%\usepackage{times}
%\usepackage{mathptm}

% this allows us to get the definitions that are not italicized

\usepackage{amsthm}

\usepackage{amsmath}
\usepackage{amsfonts}

\usepackage{url}

%\usepackage[named]{algo}
%\usepackage{algo}

\usepackage{listings}
%\usepackage{apalike}
\usepackage[authoryear]{natbib}
%\usepackage{chicago}
%\usepackage{keyval}
%\usepackage{epsfig}
\usepackage{array}
%\usepackage{fancyhdr}

% set it so that subsubsections have numbers and they
% are displayed in the TOC (maybe hard to read, might want to disable)

\setcounter{secnumdepth}{3}
\setcounter{tocdepth}{3}

% define widow protection

\def\widow#1{\vskip #1\vbadness10000\penalty-200\vskip-#1}

\clubpenalty=10000  % Don't allow orphans
\widowpenalty=10000 % Don't allow widows

% this should give me the ability to use some math symbols that
% were available by default in standard latex (i.e. \Box)

\usepackage{latexsym}

% define a little section heading that doesn't go with any number

\def\littlesection#1{
\widow{2cm}
\vskip 0.5cm
\noindent{\bf #1}
\vskip 0.0001cm
%\noindent
}

%Check if we are compiling under latex or pdflatex
%% \ifx\pdftexversion\undefined
%%   \usepackage[dvips]{graphics}
%% \else
%%   \usepackage[pdftex]{graphics}
%% \fi

%\lhead{Section \thesection}
%\chead{}
%\rhead{Page \thepage}
%\lfoot{Gregory M. Kapfhammer}
%\cfoot{}
%\rfoot{\today}

%% \pagestyle{fancyplain}

%% \newcommand{\tstamp}{\today}
%% %\renewcommand{\chaptermark}[1]{\markboth{#1}{}}
%% \renewcommand{\sectionmark}[1]{\markright{#1}}
%% \lhead[\Section \thesection]            {\fancyplain{}{\rightmark}}
%% \chead[\fancyplain{}{}]                 {\fancyplain{}{}}
%% \rhead[\fancyplain{}{\rightmark}]       {\fancyplain{}{\thepage}}
%% %\lfoot[\fancyplain{}{}]                 {\fancyplain{\tstamp}{\tstamp}}
%% \cfoot[\fancyplain{\thepage}{}]         {\fancyplain{\thepage}{}}
%% %\rfoot[\fancyplain{\tstamp} {\tstamp}]  {\fancyplain{}{}}

%% \newlength{\myVSpace}% the height of the box
%% \setlength{\myVSpace}{1ex}% the default,
%% \newcommand\xstrut{\raisebox{-.5\myVSpace}% symmetric behaviour,
%%   {\rule{0pt}{\myVSpace}}%
%% }

% A paraphrase mode that makes it easy to see the stuff that shouldn't
% stay in for the final paper

\newdimen\tmpdim
\long\def\paraphrase#1{{\parskip=0pt\hfil\break
\tmpdim=\hsize\advance\tmpdim by -15pt\noindent%
\hbox to \hsize
{\vrule\hskip 3pt\vrule\hfil\hbox to \tmpdim{\vbox{\hsize=\tmpdim
\def\par{\leavevmode\endgraf}
\obeyspaces \obeylines
\let\par=\endgraf
\bf #1}}}}}

% leave things with no spacing extra spacing in the final version of the paper
\renewcommand{\baselinestretch}{1.0}    % must go before the begin of doc

% suppress the use of indentation for a paragraph

\setlength{\parindent}{0.0in}
\setlength{\parskip}{0.1in}

\begin{document}

%\baselineskip=22pt
%\setlength{\parindent} {.5in}

\newtheorem{assume}{Assumption}

\newtheorem{consider}{Consideration}

% we are now going with the new method that is from the DIATOMS proposal

%\newtheorem{problem}{Problem}
%\newtheorem{define}{Definition}

\theoremstyle{definition}
\newtheorem{definition}{Definition}

\theoremstyle{definition}
\newtheorem{problem}{Problem}

\theoremstyle{definition}
\newtheorem{principle}{Principle}

% handle widows appropriately
\def\widow#1{\vskip #1\vbadness10000\penalty-200\vskip-#1}

% build the title section

\makeatletter

\def\maketitle{%
  %\null
  \thispagestyle{empty}%
  %\vfill
  \begin{center}%\leavevmode
    %\normalfont
    {\Huge \@title\par}%
    %\hrulefill\par
    {\normalsize \@author\par}%
    \vskip .4in
%    {\Large \@date\par}%
  \end{center}%
  %\vfill
  %\null
  %\cleardoublepage

  }

\makeatother

% build the author section
Leana Golubchik \\
Computer Science Department \\
University of Southern California \\
October 27, 2010

\noindent
Dear Editor,

My co-authors and I would like to thank the editor and past-editor of
{\em Performance Evaluation Review} and the reviewers for enabling us
to make substantial improvements to our paper.  This cover letter
outlines the changes that we made in response to the comments from the
two reviewers.  In what follows, we summarize all of the concerns of
each reviewer and then explain how we handled the matter (we sometimes
consider several related comments in the same point).

Reviewer 1: \vspace*{-.15in}
\renewcommand{\labelitemi}{$\rightarrow$}

\begin{enumerate}

  \item {\bf List of Contributions}: The paper contains a list of
    contributions that is too long. \label{item:toomany}

    \begin{itemize}

      \item We have substantially reduced the number of items listed
        as contributions.  For instance, all of the points about the
        experimental results have now been removed and placed into the
        appropriate location in Section 5.

    \end{itemize}

  \item {\bf Benchmark Naming Conventions}: The names for the
    benchmarks are misleading because the ``macro'' benchmarks are
    only slightly bigger than the ``micro'' benchmarks.

    \begin{itemize}

      \item Section 2.2 now refers to ``nano'' and ``micro''
        benchmarks, with Footnote 1 justifying our choice of this
        terminology.  All of the text and graphs have been updated to
        reflect this naming convention, which we deem to be both
        accurate and easy to understand.

    \end{itemize}

  \item {\bf Choice of Communication Primitives}: Both of the chosen
    communication primitives require access to the network interface.
    Wouldn't it be possible to use other, perhaps more efficient,
    primitives that can avoid accessing the
    network? \label{item:choice}

    \begin{itemize}

      \item Section 1 contains a revised introduction that better
        justifies our focus on remote communication primitives in the
        Java programming language.  In essence, many of the custom
        communication primitives can incur increases in source code
        complexity and implementation effort.  The empirical study in
        this paper in the first step towards determining if
        easy-to-program remote communication primitives in the Java
        language could be used instead of custom schemes.  Moreover,
        the benchmarking framework in this paper could also be used to
        evaluate other types of communication primitives.  We have
        revised Sections 1, 2, 3, 6, and 7 to reflect these facts.

    \end{itemize}

  \item {\bf Minor Changes to Graphs and Sentences}: The fonts in
    Figures 3, 4, and 5 are too small and several of the sentences are
    difficult to understand.

    \begin{itemize}

      \item In order to ensure that the paper fits into the ten page
        limit, we cut many paragraphs of content.  However, space
        constraints currently prevent us from substantially increasing
        the font size in the graphs.  Finally, we have revised many of
        the sentences in the paper in order to guarantee that the
        content is easy to understand.

    \end{itemize}

\end{enumerate}

\newpage

Reviewer 2: \vspace*{-.15in}
\renewcommand{\labelitemi}{$\rightarrow$}

\begin{enumerate}

  \item {\bf Focus and Motivation of the Paper}: Can you better
    motivate why this result would help to solve a real-world problem
    in the area of intra-node communication?

    \begin{itemize}

      \item We judge that this concern is connected to
        Item~\ref{item:choice} raised by Reviewer 1.  We modified
        several sections of the paper, including the abstract and
        Sections 1, 2, 3, 6, and 7, in order to clarify the motivation
        of this article.

    \end{itemize}

  \item {\bf Title of the Paper}: The initial title of the paper uses
    the term ``primitive'' to refer to a ``communication primitive.''
    However, both a table and a paragraph in the paper also use the
    word to stand for a ``data primitive.''

    \begin{itemize}

      \item We have resolved this naming conflict by changing the
        title of the paper and now carefully using the term ``remote
        communication primitive'' through the article.  Instead of
        using the phrase ``data primitive'' we now write ``data
        value'' in Tables 1 and 2 and throughout the paper.  Following
        the reviewer's suggestion, the new title of the paper better
        connects to the performance evaluation that we undertook.

    \end{itemize}

  \item {\bf Additional Research Questions}: Can you address other
    research questions such as ``how do the results depend on using
    different data types?''

    \begin{itemize}

      \item Our primary focus is to use the benchmarking framework to
        measure and explain response time results for two remote
        communication primitives.  In order to accomplish this task
        well, we were not able to fully address the suggested research
        questions.  However, the presented framework can be enhanced
        and Section 7 has been re-written to better highlight future
        endeavors to more fully answer your questions.

    \end{itemize}

  \item {\bf Errors in Computing the Metrics}: Can you fix the errors
    and/or discrepancies in the calculations for the percent change in
    the evaluation metrics?

    \begin{itemize}

      \item We revisited and ultimately hand-checked each of the
        calculations for every metric listed in Section 4.  The
        current version of the paper contains correct values for all
        of the metrics and the corresponding percent increase values.
        Furthermore, we carefully revised the notation in Section 4
        that explains each of the metrics.  For instance, the paper
        now fully describes the different types of percent change in
        the metrics.


    \end{itemize}

  \item {\bf Missing References}: Can you include the references to
    things like the HotSpot adaptive optimizer and certain papers that
    support your assertions?

    \begin{itemize}

      \item We now include a reference to a paper that describes the
        HotSpot adaptive optimization component. We do not include a
        specific reference for projects like XML-RPC due to space
        constraints and the fact that the Web site addresses for this
        type of project may frequently change.  We also reference a
        paper that supports our assertion that the use of XML-RPC may
        compromise performance.

    \end{itemize}

\newpage

  \item {\bf Reorganize the Introduction}: Can you remove some of the
    points that are in the list of contributions (e.g., threats to
    validity)?

    \begin{itemize}

      \item We think that this concern is connected to
        Item~\ref{item:toomany} raised by Reviewer 1.  As such, we
        have removed the discussion of threats to validity from the
        list of contributions.

    \end{itemize}

  \item {\bf Clarify Sentences and Notation}: Several of the sentences
    contain phrases that are hard to understand -- can you fix these?
    Also, the notation should be changed so that you use $N$ trials in
    the algorithm instead of $L$.

    \begin{itemize}

      \item We have made all of the requested changes in sentence
        structure and notation.  For instance, the algorithm in Figure
        2 and all of the content in the paper now use $N$ to denote
        the number of trials.  We have also clarified the meaning of
        the term ``sequence'' and, as previously mentioned, we use
        ``data value'' instead of ``data primitive.''  In summary, we
        have made all of these requested changes in Sections 1 through
        7.

    \end{itemize}

  \item {\bf Table Movement}: Can you move the tables to the same page
    where they are discussed?

    \begin{itemize}

      \item We moved Tables 1 and 2 to the third page of the paper.
        We tried to keep tables and figures close to the appropriate
        paragraphs whenever this layout would not ruin the overall
        organization and aesthetic appeal of the paper.

    \end{itemize}

  \item {\bf Benchmark Name}: Can you change the name of a benchmark to
    better reflect its actual purpose?  Furthermore, the paper must
    clearly define the meaning of $B$ and $P$.

    \begin{itemize}

      \item Instead of using the term ``FIND'' we now write ``GRAB''
        to reinforce the fact that the algorithm is $O(1)$.  We have
        made this change in both the text and the graphs.  The paper
        now explicitly states that $B$ can be any one of the
        benchmarks listed in Tables 1 and 2 and that $P$ can stand for
        either sockets ($S$) or XML-RPC ($X$).

    \end{itemize}

  \item {\bf Statistical Assumptions}: Even though it only directly
    refers to two ``assumptions,'' the paper defines three that
    undergird the statistical analysis.

    \begin{itemize}

      \item Instead of using the term ``assumption'' we now write
        ``observation.'' We re-wrote the paragraph in Section
        4.2 so that it refers to all three of the observations.

    \end{itemize}

  \item {\bf String Processing}: The paper states that the XML-RPC
    server must perform ``additional string processing'' -- can you
    explain why this is the case?

    \begin{itemize}

      \item Section 5.3 now contains a full paragraph to explain more
        about how the implementation of the benchmarks impacts the
        performance results.  Briefly, both the socket and XML-RPC
        benchmarks use strings and character data at least one time
        during the transmission of the parameters and the return
        values.  We also cite reference [21] in support of the claim
        that additional string processing may cause the benchmarks to
        inefficiently allocate data values to the heap.

    \end{itemize}

\newpage

  \item {\bf Modify the Conclusions and Future Work}: Some of the
    material in the related work section is not absolutely needed
    (e.g., ``Performance Benchmarks'' and ``Java Performance
    Analysis'').  Furthermore, several of the sentences are written in
    a confusing fashion.

    \begin{itemize}

      \item We re-wrote Section 6 so that it focuses on related work
        in the areas of communication primitives and sockets and
        XML-RPC.  The current version of the paper also avoids the
        confusing use of ``(i)'', ``(ii)'', etc.

    \end{itemize}

  \item {\bf Explanatory Nature of Tables and Figures}: All of the
    tables and figures should be self-contained in nature.

    \begin{itemize}

      \item We added one line comments to Tables 4, 5, and 7 in order
        to ensure that the reader can understand the content without
        having to read all of the surrounding content.  For instance,
        we now clarify that Table 4 only refers to response times
        calculated with the Java language-based timers.

    \end{itemize}

  \item {\bf Differentiate the Timers}: A cursory glance at Table 5
    does not allow the reader to determine what type of timer was
    used.

    \begin{itemize}

      \item The final column in Table 5 now contains a checkmark to
        indicate which type of instrumentation was used to collect the
        response time values.  Furthermore, we carefully checked that
        the text in Section 5 refers to each one of the rows in this
        table.

    \end{itemize}

  \item {\bf Discuss the Subscripts}: The paper should explain the
    meaning of notation like $S_{500}$.

    \begin{itemize}

      \item Footnote 3 in Section 5 now explains the meaning of the
        notation $P_k$, using $S_{500}$ as a concrete example that
        stands for the use of sockets to transmit a vector of size
        500.

    \end{itemize}

  \item {\bf Table Headers}: All table headers should fully explain
    the data values in the column.

    \begin{itemize}

      \item We carefully checked to ensure that all of the headers in
        the table accurately reflect the data values.  For instance,
        the header in Table 7 now indicates that the values correspond
        to a count of the number of packets.

    \end{itemize}

\end{enumerate}

\vspace*{-.15in}

In addition to making all of the suggested changes (and those
modifications implied by these requests), we also revised the
algorithm in Figure 2 and the equations in Section 4.  For instance,
Equation (5) and the preceding paragraph clearly explain how the
benchmarking framework estimates the total network consumption in
bytes.  The paper now explains how we calculate the percent changes
across a group of benchmarks and it gives the equations for computing
the arithmetic mean and the standard deviations.  We have also
enhanced all of the sentences and paragraphs in the paper.  As an
example, the new abstract is both more focused and shorter than the
one in the previous edition.  In conclusion, we hope that this
revised paper handles the reviewers' concerns and we look forward to
soon learning more about the status of the article.

\vspace*{.15in}
\noindent
Gregory M. Kapfhammer \\
Department of Computer Science \\
Allegheny  College

\end{document}
